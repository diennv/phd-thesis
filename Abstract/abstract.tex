% ************************** Thesis Abstract *****************************
% Use `abstract' as an option in the document class to print only the titlepage and the abstract.
%Every academic paper begins with title page. Its structure depends on the chosen formatting style. An abstract follows it. This is an important part that describes thesis utility. It must be short and take 1-2 paragraphs, about 400 words and contain short summary of results, methods, etc. Here are questions to answer in this part:

%What was the reason to write this paper?
%What thesis statement to prove or disprove?
%What were your instruments? (describe main methods of research)
%What did you find out?
%Why are the results important?
\begin{abstract}
Video Analytics(VA) systems have an important role in many security areas such as public security, transportation and other industry fields. An important feature of these platform which is difficult and challenging to analyze at “real-time” speed to generate periodly event notifications with different scenario application. Because of the bandwidth and computation limitation, VA servers can not meet the “real-time” requirements when serving for many cameras simultaneously. In order to overcome these challenges, this dissertation propose a method to minimizes transmit of camera frame from sources to a video analytics server on the cloud by using an edge-computing based system. The proposed method was conducted and published in our lab publication \cite{nguyen2020toward}. By extracting the feature from the camera bitstream, our edge device obtain low-complexity of the moving object detection algorithm to filter uninteresting frames from sender buffer which will be forwarded to the video analytics server. In order to show the performance of the proposed solution, we deployed a camera based surveillance system which consist of a low computational edge device and a hight computing server with GPU supported. The performance results indicate that our approach is compatible with the “edge-to-cloud” platform and  can efficiently catch the motion of frame. The average inference time of this approach is around 39 miliseconds per frame with high definition(HD) quality, which is the state of art method. The experiment results show that the proposed method makes the cloud server reduce 48\% utilization of GPU, 48\% utilization of CPU, and 51\% of download throughput while still keeping the accuracy of real-time alerts.
\end{abstract}
